% \documentclass{ctexart}
% \usepackage{xchoices}
% \usepackage{graphicx}
% % \usepackage[a4paper]{geometry}
% % \xchoicesetup{
% %   showanswer,
% %   showanalysis
% % }
% \begin{document}
%     % \begin{xchoices}
%     %   \item
%     %   $3 n(\sin \frac{30^{\circ}}{n}+\tan \frac{30^{\circ}}{n})$
%     %   \item*
%     %   $6 n(\sin \frac{30^{\circ}}{n}+\tan \frac{30^{\circ}}{n})$
%     %   \item*
%     %   $3 n(\sin \frac{60^{\circ}}{n}+\tan \frac{30^{\circ}}{n})$
%     %   \item
%     %   $6 n(\sin \frac{30^{\circ}}{n}+\tan \frac{30^{\circ}}{n})$ 
%     % \end{xchoices}
%   \xchoicesetup{ label-pos = left }
%   2020 年 3 月 14 日是全球首个国际圆周率日 ( $\pi$ Day). 历史上, 求圆周率 $\pi$ 的方法有多种, 与中国传统数学中的 “割圆术”相似. 数学家阿尔. 卡西的方 法是: 当正整数 $n$ 充分大时, 计算单位圆的内接正 $6 n$ 边形的周长和外切 正 $6 n$ 边形 (各边均与圆相切的正 $6 n$ 边形) 的周长, 将它们的算术平均数 作为 $2 \pi$ 的近似值. 按照阿尔. 卡西的方法, $\pi$ 的近似值的表达式是 \paren

%     \begin{xchoices}[items = 1]
%       \item testtest
%       \item testtesttesttest
%       \item testtesttesttesttest
%       \item testtesttesttesttesttestte
%       \item testtesttesttesttesttesttesttes
%     \end{xchoices}

%     2020 年 3 月 14 日是全球首个国际圆周率日 ( $\pi$ Day). 历史上, 求圆周率 $\pi$ 的方法有多种, 与中国传统数学中的 “割圆术”相似. 数学家阿尔. 卡西的方 法是: 当正整数 $n$ 充分大时, 计算单位圆的内接正 $6 n$ 边形的周长和外切 正 $6 n$ 边形 (各边均与圆相切的正 $6 n$ 边形) 的周长, 将它们的算术平均数 作为 $2 \pi$ 的近似值. 按照阿尔. 卡西的方法, $\pi$ 的近似值的表达式是 \paren

%     \begin{xchoices}[items = 2]
%       \item testtest
%       \item testtesttesttest
%       \item testtesttesttesttest
%       \item testtesttesttesttesttestte
%       \item testtesttesttesttesttesttesttes
%     \end{xchoices}

%     2020 年 3 月 14 日是全球首个国际圆周率日 ( $\pi$ Day). 历史上, 求圆周率 $\pi$ 的方法有多种, 与中国传统数学中的 “割圆术”相似. 数学家阿尔. 卡西的方 法是: 当正整数 $n$ 充分大时, 计算单位圆的内接正 $6 n$ 边形的周长和外切 正 $6 n$ 边形 (各边均与圆相切的正 $6 n$ 边形) 的周长, 将它们的算术平均数 作为 $2 \pi$ 的近似值. 按照阿尔. 卡西的方法, $\pi$ 的近似值的表达式是 \paren

%     \begin{xchoices}[items = 3]
%       \item testtest
%       \item testtesttesttest
%       \item testtesttesttesttest
%       \item testtesttesttesttesttestte
%       \item testtesttesttesttesttesttesttes
%     \end{xchoices}

%     2020 年 3 月 14 日是全球首个国际圆周率日 ( $\pi$ Day). 历史上, 求圆周率 $\pi$ 的方法有多种, 与中国传统数学中的 “割圆术”相似. 数学家阿尔. 卡西的方 法是: 当正整数 $n$ 充分大时, 计算单位圆的内接正 $6 n$ 边形的周长和外切 正 $6 n$ 边形 (各边均与圆相切的正 $6 n$ 边形) 的周长, 将它们的算术平均数 作为 $2 \pi$ 的近似值. 按照阿尔. 卡西的方法, $\pi$ 的近似值的表达式是 \paren

%     \begin{xchoices}[items = 4]
%       \item testtest
%       \item testtesttesttest
%       \item testtesttesttesttest
%       \item testtesttesttesttesttestte
%       \item testtesttesttesttesttesttesttes
%     \end{xchoices}

%     2020 年 3 月 14 日是全球首个国际圆周率日 ( $\pi$ Day). 历史上, 求圆周率 $\pi$ 的方法有多种, 与中国传统数学中的 “割圆术”相似. 数学家阿尔. 卡西的方 法是: 当正整数 $n$ 充分大时, 计算单位圆的内接正 $6 n$ 边形的周长和外切 正 $6 n$ 边形 (各边均与圆相切的正 $6 n$ 边形) 的周长, 将它们的算术平均数 作为 $2 \pi$ 的近似值. 按照阿尔. 卡西的方法, $\pi$ 的近似值的表达式是 \paren

%     \begin{xchoices}[items = 5]
%       \item testtest
%       \item testtesttesttest
%       \item testtesttesttesttest
%       \item testtesttesttesttesttestte
%       \item testtesttesttesttesttesttesttes
%     \end{xchoices}

%     2020 年 3 月 14 日是全球首个国际圆周率日 ( $\pi$ Day). 历史上, 求圆周率 $\pi$ 的方法有多种, 与中国传统数学中的 “割圆术”相似. 数学家阿尔. 卡西的方 法是: 当正整数 $n$ 充分大时, 计算单位圆的内接正 $6 n$ 边形的周长和外切 正 $6 n$ 边形 (各边均与圆相切的正 $6 n$ 边形) 的周长, 将它们的算术平均数 作为 $2 \pi$ 的近似值. 按照阿尔. 卡西的方法, $\pi$ 的近似值的表达式是 \paren

%     \begin{xchoices}[mode = figure, items = 1]
%       \item* 正确选项
%       \item  \includegraphics[width = 2cm]{example-image-a}
%       \item* 正确选项
%       \item  错误选项
%       \item  \includegraphics[width = 2cm]{example-image-a}
%       \item  \includegraphics[width = 2cm]{example-image-a}
%       \item  \includegraphics[width = 2cm]{example-image-a}
%       \item  错误选项
%     \end{xchoices}

%     2020 年 3 月 14 日是全球首个国际圆周率日 ( $\pi$ Day). 历史上, 求圆周率 $\pi$ 的方法有多种, 与中国传统数学中的 “割圆术”相似. 数学家阿尔. 卡西的方 法是: 当正整数 $n$ 充分大时, 计算单位圆的内接正 $6 n$ 边形的周长和外切 正 $6 n$ 边形 (各边均与圆相切的正 $6 n$ 边形) 的周长, 将它们的算术平均数 作为 $2 \pi$ 的近似值. 按照阿尔. 卡西的方法, $\pi$ 的近似值的表达式是 \paren

%     \begin{xchoices}[mode = figure, items = 2]
%       \item* 正确选项
%       \item  \includegraphics[width = 2cm]{example-image-a}
%       \item* 正确选项
%       \item  错误选项
%       \item  \includegraphics[width = 2cm]{example-image-a}
%       \item  \includegraphics[width = 2cm]{example-image-a}
%       \item  \includegraphics[width = 2cm]{example-image-a}
%       \item  错误选项
%     \end{xchoices}

%     2020 年 3 月 14 日是全球首个国际圆周率日 ( $\pi$ Day). 历史上, 求圆周率 $\pi$ 的方法有多种, 与中国传统数学中的 “割圆术”相似. 数学家阿尔. 卡西的方 法是: 当正整数 $n$ 充分大时, 计算单位圆的内接正 $6 n$ 边形的周长和外切 正 $6 n$ 边形 (各边均与圆相切的正 $6 n$ 边形) 的周长, 将它们的算术平均数 作为 $2 \pi$ 的近似值. 按照阿尔. 卡西的方法, $\pi$ 的近似值的表达式是 \paren

%     \begin{xchoices}[mode = figure, items = 3]
%       \item* 正确选项
%       \item  \includegraphics[width = 2cm]{example-image-a}
%       \item* 正确选项
%       \item  错误选项
%       \item  \includegraphics[width = 2cm]{example-image-a}
%       \item  \includegraphics[width = 2cm]{example-image-a}
%       \item  \includegraphics[width = 2cm]{example-image-a}
%       \item  错误选项
%     \end{xchoices}

%     2020 年 3 月 14 日是全球首个国际圆周率日 ( $\pi$ Day). 历史上, 求圆周率 $\pi$ 的方法有多种, 与中国传统数学中的 “割圆术”相似. 数学家阿尔. 卡西的方 法是: 当正整数 $n$ 充分大时, 计算单位圆的内接正 $6 n$ 边形的周长和外切 正 $6 n$ 边形 (各边均与圆相切的正 $6 n$ 边形) 的周长, 将它们的算术平均数 作为 $2 \pi$ 的近似值. 按照阿尔. 卡西的方法, $\pi$ 的近似值的表达式是 \paren

%     \begin{xchoices}[mode = figure, items = 4]
%       \item* 正确选项
%       \item  \includegraphics[width = 2cm]{example-image-a}
%       \item* 正确选项
%       \item  错误选项
%       \item  \includegraphics[width = 2cm]{example-image-a}
%       \item  \includegraphics[width = 2cm]{example-image-a}
%       \item  \includegraphics[width = 2cm]{example-image-a}
%       \item  错误选项
%     \end{xchoices}

%     2020 年 3 月 14 日是全球首个国际圆周率日 ( $\pi$ Day). 历史上, 求圆周率 $\pi$ 的方法有多种, 与中国传统数学中的 “割圆术”相似. 数学家阿尔. 卡西的方 法是: 当正整数 $n$ 充分大时, 计算单位圆的内接正 $6 n$ 边形的周长和外切 正 $6 n$ 边形 (各边均与圆相切的正 $6 n$ 边形) 的周长, 将它们的算术平均数 作为 $2 \pi$ 的近似值. 按照阿尔. 卡西的方法, $\pi$ 的近似值的表达式是 \paren

%     \begin{xchoices}[mode = figure, items = 5]
%       \item* 正确选项
%       \item  \includegraphics[width = 2cm]{example-image-a}
%       \item* 正确选项
%       \item  错误选项
%       \item  \includegraphics[width = 2cm]{example-image-a}
%       \item  \includegraphics[width = 2cm]{example-image-a}
%       \item  \includegraphics[width = 2cm]{example-image-a}
%       \item  错误选项
%     \end{xchoices}
% \end{document}
\documentclass{ctexart}
\usepackage{xchoices}
\fboxsep=0pt
\begin{document}

\begin{enumerate}
  \item 
  \begin{xchoices}[label-style = chinese, showanswer]
    \item* \fbox{Lorem ipsum.}
    \item \fbox{norem opsum.}
  \end{xchoices}
  \item 666
\end{enumerate}

% \begin{xchoices}[label-style = Roman, label-pos = left, showanswer]
%   \item Lorem ipsum.
%   \item norem opsum. norem opsum. norem opsum. norem opsum. norem opsum.
% \end{xchoices}

test \parbox[b]{3cm}{Lorem ipsum.}
\end{document}