\documentclass{ctexart}
\usepackage{xchoices}
\usepackage{graphicx}
\usepackage[a4paper]{geometry}
% \xchoicesetup{
%   showanswer,
%   showanalysis
% }
\begin{document}
% \begin{enumerate}
%   \item 2020 年 3 月 14 日是全球首个国际圆周率日 ( $\pi$ Day). 历史上, 求圆周率 $\pi$ 的方法有多种, 与中国传统数学中的 “割圆术”相似. 数学家阿尔. 卡西的方 法是: 当正整数 $n$ 充分大时, 计算单位圆的内接正 $6 n$ 边形的周长和外切 正 $6 n$ 边形 (各边均与圆相切的正 $6 n$ 边形) 的周长, 将它们的算术平均数 作为 $2 \pi$ 的近似值. 按照阿尔. 卡西的方法, $\pi$ 的近似值的表达式是 \paren
%     \begin{xchoice}[
%       label-style = none,
%       analysis-content = {
%         It's obvious that A is false, then ...
%       }
%     ]
%       \xitem
%       $3 n(\sin \frac{30^{\circ}}{n}+\tan \frac{30^{\circ}}{n})$
%       \xitem*
%       $6 n(\sin \frac{30^{\circ}}{n}+\tan \frac{30^{\circ}}{n})$
%       \xitem
%       $3 n(\sin \frac{60^{\circ}}{n}+\tan \frac{30^{\circ}}{n})$
%       \xitem
%       $6 n(\sin \frac{30^{\circ}}{n}+\tan \frac{30^{\circ}}{n})$ 
%     \end{xchoice}
  % \item 
  2020 年 3 月 14 日是全球首个国际圆周率日 ( $\pi$ Day). 历史上, 求圆周率 $\pi$ 的方法有多种, 与中国传统数学中的 “割圆术”相似. 数学家阿尔. 卡西的方 法是: 当正整数 $n$ 充分大时, 计算单位圆的内接正 $6 n$ 边形的周长和外切 正 $6 n$ 边形 (各边均与圆相切的正 $6 n$ 边形) 的周长, 将它们的算术平均数 作为 $2 \pi$ 的近似值. 按照阿尔. 卡西的方法, $\pi$ 的近似值的表达式是 \paren
    \begin{xchoices*}[
      label-pos = top, items = 2, v-sep = 0em,
      h-offset = 6em, v-offset = 0em, showanswer
    ]
      \xitem*
      $3 n(\sin \frac{30^{\circ}}{n}+\tan \frac{30^{\circ}}{n})$
      \xitem
      $6 n(\sin \frac{30^{\circ}}{n}+\tan \frac{30^{\circ}}{n})$
      \xitem*
      $3 n(\sin \frac{60^{\circ}}{n}+\tan \frac{30^{\circ}}{n})$
      \xitem
      $6 n(\sin \frac{30^{\circ}}{n}+\tan \frac{30^{\circ}}{n})$ 
    \end{xchoices*}
    2020 年 3 月 14 日是全球首个国际圆周率日 ( $\pi$ Day). 历史上, 求圆周率 $\pi$ 的方法有多种, 与中国传统数学中的 “割圆术”相似. 数学家阿尔. 卡西的方 法是: 当正整数 $n$ 充分大时, 计算单位圆的内接正 $6 n$ 边形的周长和外切 正 $6 n$ 边形 (各边均与圆相切的正 $6 n$ 边形) 的周长, 将它们的算术平均数 作为 $2 \pi$ 的近似值. 按照阿尔. 卡西的方法, $\pi$ 的近似值的表达式是 \paren
    \begin{xchoices*}[
      label-pos = top, items = 2, v-sep = 0em,
      h-offset = 6em, v-offset = 0em, showanswer
    ]
      \xitem
      $3 n(\sin \frac{30^{\circ}}{n}+\tan \frac{30^{\circ}}{n})$
      \xitem*
      $6 n(\sin \frac{30^{\circ}}{n}+\tan \frac{30^{\circ}}{n})$
      \xitem*
      $3 n(\sin \frac{60^{\circ}}{n}+\tan \frac{30^{\circ}}{n})$
      \xitem
      $6 n(\sin \frac{30^{\circ}}{n}+\tan \frac{30^{\circ}}{n})$ 
    \end{xchoices*}
% \end{enumerate}
% \xchoices{
%   正确选项 &&
%   正确选项 &&
%   正确选项 &&
%   正确选项正确选项正确选项正确选项正确选项
% }
% % \begin{hlist}1
% %   \hitem 正确选项
% %   \hitem  错误选项
% %   \hitem 正确选项
% %   \hitem  错误选项错误选项错误选项错误选项错误选项错误选项
% % \end{hlist}
% \begin{xchoice}[
%   % label-style = quan, showanswer,
%   analysis-content = {
%     It's obvious that D is false, then ...
%   }
% ]
%   \xitem* 正确选项
%   \xitem  错误选项
%   \xitem* 正确选项
%   \xitem  错误选项错误选项错误选项错误选项错误选项错误选项
% \end{xchoice}

% 2020 年 3 月 14 日是全球首个国际圆周率日 ( $\pi$ Day). 历史上, 求圆周率 $\pi$ 的方法有多种, 与中国传统数学中的 “割圆术”相似. 数学家阿尔. 卡西的方 法是: 当正整数 $n$ 充分大时, 计算单位圆的内接正 $6 n$ 边形的周长和外切 正 $6 n$ 边形 (各边均与圆相切的正 $6 n$ 边形) 的周长, 将它们的算术平均数 作为 $2 \pi$ 的近似值. 按照阿尔. 卡西的方法, $\pi$ 的近似值的表达式是 \paren

% \xchoices*[label-pos = top-left, items = 1]{
%   1111 &&
%   2222 &&
%   \includegraphics[width = 3cm]{example-image-a} &&
%   3333 &&
%   444444 &&
%   \includegraphics[width = 3cm]{example-image-a} &&
%   55555  &&
%   666666 &&
%   \includegraphics[width = 3cm]{example-image-a} &&
%   \includegraphics[width = 3cm]{example-image-a} &&
% }

% 2020 年 3 月 14 日是全球首个国际圆周率日 ( $\pi$ Day). 历史上, 求圆周率 $\pi$ 的方法有多种, 与中国传统数学中的 “割圆术”相似. 数学家阿尔. 卡西的方 法是: 当正整数 $n$ 充分大时, 计算单位圆的内接正 $6 n$ 边形的周长和外切 正 $6 n$ 边形 (各边均与圆相切的正 $6 n$ 边形) 的周长, 将它们的算术平均数 作为 $2 \pi$ 的近似值. 按照阿尔. 卡西的方法, $\pi$ 的近似值的表达式是 \paren

% \xchoices*[label-pos = top-left, items = 2]{
%   1111 &&
%   2222 &&
%   \includegraphics[width = 3cm]{example-image-a} &&
%   3333 &&
%   444444 &&
%   \includegraphics[width = 3cm]{example-image-a} &&
%   55555  &&
%   666666 &&
%   \includegraphics[width = 3cm]{example-image-a} &&
%   \includegraphics[width = 3cm]{example-image-a} &&
% }

% 2020 年 3 月 14 日是全球首个国际圆周率日 ( $\pi$ Day). 历史上, 求圆周率 $\pi$ 的方法有多种, 与中国传统数学中的 “割圆术”相似. 数学家阿尔. 卡西的方 法是: 当正整数 $n$ 充分大时, 计算单位圆的内接正 $6 n$ 边形的周长和外切 正 $6 n$ 边形 (各边均与圆相切的正 $6 n$ 边形) 的周长, 将它们的算术平均数 作为 $2 \pi$ 的近似值. 按照阿尔. 卡西的方法, $\pi$ 的近似值的表达式是 \paren

% \xchoices*[label-pos = top-left, items = 3]{
%   1111 &&
%   2222 &&
%   \includegraphics[width = 3cm]{example-image-a} &&
%   3333 &&
%   444444 &&
%   \includegraphics[width = 3cm]{example-image-a} &&
%   55555  &&
%   666666 &&
%   \includegraphics[width = 3cm]{example-image-a} &&
%   \includegraphics[width = 3cm]{example-image-a} &&
% }

% \newpage
% 2020 年 3 月 14 日是全球首个国际圆周率日 ( $\pi$ Day). 历史上, 求圆周率 $\pi$ 的方法有多种, 与中国传统数学中的 “割圆术”相似. 数学家阿尔. 卡西的方 法是: 当正整数 $n$ 充分大时, 计算单位圆的内接正 $6 n$ 边形的周长和外切 正 $6 n$ 边形 (各边均与圆相切的正 $6 n$ 边形) 的周长, 将它们的算术平均数 作为 $2 \pi$ 的近似值. 按照阿尔. 卡西的方法, $\pi$ 的近似值的表达式是 \paren


% \xchoices*[label-pos = top-left, items = 4]{
%   1111 &&
%   2222 &&
%   \includegraphics[width = 3cm]{example-image-a} &&
%   3333 &&
%   444444 &&
%   \includegraphics[width = 3cm]{example-image-a} &&
%   55555  &&
%   666666 &&
%   \includegraphics[width = 3cm]{example-image-a} &&
%   \includegraphics[width = 3cm]{example-image-a} &&
% }

% \newpage

% 2020 年 3 月 14 日是全球首个国际圆周率日 ( $\pi$ Day). 历史上, 求圆周率 $\pi$ 的方法有多种, 与中国传统数学中的 “割圆术”相似. 数学家阿尔. 卡西的方 法是: 当正整数 $n$ 充分大时, 计算单位圆的内接正 $6 n$ 边形的周长和外切 正 $6 n$ 边形 (各边均与圆相切的正 $6 n$ 边形) 的周长, 将它们的算术平均数 作为 $2 \pi$ 的近似值. 按照阿尔. 卡西的方法, $\pi$ 的近似值的表达式是 \paren
% \xchoices*[label-pos = top-left, items = 1]{
%   1111 &&
%   2222 &&
%   % \includegraphics[width = 3cm]{example-image-a} &&
%   3333 &&
%   444444 &&
%   % \includegraphics[width = 3cm]{example-image-a} &&
%   55555  &&
%   666666 &&
%   % \includegraphics[width = 3cm]{example-image-a} &&
%   % \includegraphics[width = 3cm]{example-image-a} &&
% }
% 2020 年 3 月 14 日是全球首个国际圆周率日 ( $\pi$ Day). 历史上, 求圆周率 $\pi$ 的方法有多种, 与中国传统数学中的 “割圆术”相似. 数学家阿尔. 卡西的方 法是: 当正整数 $n$ 充分大时, 计算单位圆的内接正 $6 n$ 边形的周长和外切 正 $6 n$ 边形 (各边均与圆相切的正 $6 n$ 边形) 的周长, 将它们的算术平均数 作为 $2 \pi$ 的近似值. 按照阿尔. 卡西的方法, $\pi$ 的近似值的表达式是 \paren
% \xchoices*[label-pos = top-left, items = 2]{
%   1111 &&
%   2222 &&
%   % \includegraphics[width = 3cm]{example-image-a} &&
%   3333 &&
%   444444 &&
%   % \includegraphics[width = 3cm]{example-image-a} &&
%   55555  &&
%   666666 &&
%   % \includegraphics[width = 3cm]{example-image-a} &&
%   % \includegraphics[width = 3cm]{example-image-a} &&
% }
% 2020 年 3 月 14 日是全球首个国际圆周率日 ( $\pi$ Day). 历史上, 求圆周率 $\pi$ 的方法有多种, 与中国传统数学中的 “割圆术”相似. 数学家阿尔. 卡西的方 法是: 当正整数 $n$ 充分大时, 计算单位圆的内接正 $6 n$ 边形的周长和外切 正 $6 n$ 边形 (各边均与圆相切的正 $6 n$ 边形) 的周长, 将它们的算术平均数 作为 $2 \pi$ 的近似值. 按照阿尔. 卡西的方法, $\pi$ 的近似值的表达式是 \paren
% \xchoices*[label-pos = top-left, items = 3]{
%   1111 &&
%   2222 &&
%   % \includegraphics[width = 3cm]{example-image-a} &&
%   3333 &&
%   444444 &&
%   % \includegraphics[width = 3cm]{example-image-a} &&
%   55555  &&
%   666666 &&
%   % \includegraphics[width = 3cm]{example-image-a} &&
%   % \includegraphics[width = 3cm]{example-image-a} &&
% }
% 2020 年 3 月 14 日是全球首个国际圆周率日 ( $\pi$ Day). 历史上, 求圆周率 $\pi$ 的方法有多种, 与中国传统数学中的 “割圆术”相似. 数学家阿尔. 卡西的方 法是: 当正整数 $n$ 充分大时, 计算单位圆的内接正 $6 n$ 边形的周长和外切 正 $6 n$ 边形 (各边均与圆相切的正 $6 n$ 边形) 的周长, 将它们的算术平均数 作为 $2 \pi$ 的近似值. 按照阿尔. 卡西的方法, $\pi$ 的近似值的表达式是 \paren
% \xchoices*[label-pos = top-left, items = 5]{
%   1111 &&
%   2222 &&
%   3333 &&
%   444444 &&
%   55555  &&
%   666666 &&
%   1111 &&
%   2222 &&
%   3333 &&
%   444444 &&
%   55555  &&
%   666666 &&
%   1111 &&
%   2222 &&
%   3333 &&
%   444444 &&
%   55555  &&
%   666666 &&
% }
\end{document}