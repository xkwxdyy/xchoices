\documentclass{ctexart}
\usepackage{xchoices}
\usepackage{graphicx}
\usepackage[a4paper]{geometry}
% \xchoicesetup{
%   showanswer,
%   showanalysis
% }
\begin{document}
% \begin{enumerate}
%   \item 2020 年 3 月 14 日是全球首个国际圆周率日 ( $\pi$ Day). 历史上, 求圆周率 $\pi$ 的方法有多种, 与中国传统数学中的 “割圆术”相似. 数学家阿尔. 卡西的方 法是: 当正整数 $n$ 充分大时, 计算单位圆的内接正 $6 n$ 边形的周长和外切 正 $6 n$ 边形 (各边均与圆相切的正 $6 n$ 边形) 的周长, 将它们的算术平均数 作为 $2 \pi$ 的近似值. 按照阿尔. 卡西的方法, $\pi$ 的近似值的表达式是 \paren
    \begin{xchoices}[showanswer]
      \item
      $3 n(\sin \frac{30^{\circ}}{n}+\tan \frac{30^{\circ}}{n})$
      \item*
      $6 n(\sin \frac{30^{\circ}}{n}+\tan \frac{30^{\circ}}{n})$
      \item*
      $3 n(\sin \frac{60^{\circ}}{n}+\tan \frac{30^{\circ}}{n})$
      \item
      $6 n(\sin \frac{30^{\circ}}{n}+\tan \frac{30^{\circ}}{n})$ 
    \end{xchoices}

    <题干内容>,下面正确的是\paren
    \begin{xchoices}[
      label-style = quan
    ]
      \item* 正确选项
      \item  \includegraphics[width = 2cm]{example-image-a}
      \item* 正确选项
      \item  错误选项
    \end{xchoices}
\end{document}