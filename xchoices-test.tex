\documentclass{ctexart}
\usepackage{xchoices}

\xchoicesetup{
  showanswer,
  showanalysis
}
\begin{document}
\begin{enumerate}
  \item 2020 年 3 月 14 日是全球首个国际圆周率日 ( $\pi$ Day). 历史上, 求圆周率 $\pi$ 的方法有多种, 与中国传统数学中的 “割圆术”相似. 数学家阿尔. 卡西的方 法是: 当正整数 $n$ 充分大时, 计算单位圆的内接正 $6 n$ 边形的周长和外切 正 $6 n$ 边形 (各边均与圆相切的正 $6 n$ 边形) 的周长, 将它们的算术平均数 作为 $2 \pi$ 的近似值. 按照阿尔. 卡西的方法, $\pi$ 的近似值的表达式是 \paren
    \begin{xchoice}[
      analysis-content = {
        It's obvious that A is false, then ...
      }
    ]
      \xitem
      $3 n(\sin \frac{30^{\circ}}{n}+\tan \frac{30^{\circ}}{n})$
      \xitem
      $6 n(\sin \frac{30^{\circ}}{n}+\tan \frac{30^{\circ}}{n})$
      \xitem*
      $3 n(\sin \frac{60^{\circ}}{n}+\tan \frac{30^{\circ}}{n})$
      \xitem
      $6 n(\sin \frac{30^{\circ}}{n}+\tan \frac{30^{\circ}}{n})$ 
    \end{xchoice}
  \item 2020 年 3 月 14 日是全球首个国际圆周率日 ( $\pi$ Day). 历史上, 求圆周率 $\pi$ 的方法有多种, 与中国传统数学中的 “割圆术”相似. 数学家阿尔. 卡西的方 法是: 当正整数 $n$ 充分大时, 计算单位圆的内接正 $6 n$ 边形的周长和外切 正 $6 n$ 边形 (各边均与圆相切的正 $6 n$ 边形) 的周长, 将它们的算术平均数 作为 $2 \pi$ 的近似值. 按照阿尔. 卡西的方法, $\pi$ 的近似值的表达式是 \paren
    \begin{xchoice}[
      analysis-content = {
        It's obvious that D is false, then ...
      }
    ]
      \xitem*
      $3 n(\sin \frac{30^{\circ}}{n}+\tan \frac{30^{\circ}}{n})$
      \xitem
      $6 n(\sin \frac{30^{\circ}}{n}+\tan \frac{30^{\circ}}{n})$
      \xitem*
      $3 n(\sin \frac{60^{\circ}}{n}+\tan \frac{30^{\circ}}{n})$
      \xitem
      $6 n(\sin \frac{30^{\circ}}{n}+\tan \frac{30^{\circ}}{n})$ 
    \end{xchoice}
\end{enumerate}

\begin{xchoice}[
  analysis-content = {
    It's obvious that D is false, then ...
  }
]
  \xitem* 正确选项
  \xitem  错误选项
  \xitem* 正确选项
  \xitem  错误选项
\end{xchoice}

\end{document}