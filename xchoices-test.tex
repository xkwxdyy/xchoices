\documentclass{ctexart}
\usepackage{xchoices}
\usepackage{graphbox}

% 统一控制答案的显示
\xchoicesetup{
  label-pos = left-bottom
}
\begin{document}
  \begin{enumerate}
    \item 
      2020 年 3 月 14 日是全球首个国际圆周率日 ( $\pi$ Day). 历史上, 求圆周率 $\pi$ 的方法有多种, 与中国传统数学中的 “割圆术”相似. 数学家阿尔. 卡西的方 法是: 当正整数 $n$ 充分大时, 计算单位圆的内接正 $6 n$ 边形的周长和外切 正 $6 n$ 边形 (各边均与圆相切的正 $6 n$ 边形) 的周长, 将它们的算术平均数 作为 $2 \pi$ 的近似值. 按照阿尔. 卡西的方法, $\pi$ 的近似值的表达式是 \xparen
        \begin{xchoices}
          \item testtest
          \item* \includegraphics[height = 10mm]{example-image-a}
          \item testtesttesttesttest
          \item* testtesttesttesttesttestte
          \item testtesttesttesttesttesttesttes
        \end{xchoices}
    \item
      2020 年 3 月 14 日是全球首个国际圆周率日 ( $\pi$ Day). 历史上, 求圆周率 $\pi$ 的方法有多种, 与中国传统数学中的 “割圆术”相似. 数学家阿尔. 卡西的方 法是: 当正整数 $n$ 充分大时, 计算单位圆的内接正 $6 n$ 边形的周长和外切 正 $6 n$ 边形 (各边均与圆相切的正 $6 n$ 边形) 的周长, 将它们的算术平均数 作为 $2 \pi$ 的近似值. 按照阿尔. 卡西的方法, $\pi$ 的近似值的表达式是 \xparen
        \begin{xchoices}[showanswer = false]
          \item 选项1
          \item* 选项2
          \item 选项3
          \item* 选项4
        \end{xchoices}
    \item  % 测试基线对齐效果
      \begin{xchoices}[label-style = alph]
        \item \textbf{ \huge norem opsum}
        \item norem opsum
        \item Lipnorem opsum
      \end{xchoices}
  \end{enumerate}
  
 
\end{document}