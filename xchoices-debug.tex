% \documentclass{ctexart}
% \usepackage{xchoices}
% % \usepackage[a3paper]{geometry}
% \usepackage[a4paper]{geometry}

% \begin{document}


% \section{图文}

% \xchoicesetup{
%   label-pos = right-bottom,
% }
% \begin{enumerate}
%   \item 2020 年 3 月 14 日是全球首个国际圆周率日 ( $\pi$ Day). 历史上, 求圆周率 $\pi$ 的方法有多种, 与中国传统数学中的 “割圆术”相似. 数学家阿尔. 卡西的方 法是: 当正整数 $n$ 充分大时, 计算单位圆的内接正 $6 n$ 边形的周长和外切 正 $6 n$ 边形 (各边均与圆相切的正 $6 n$ 边形) 的周长, 将它们的算术平均数 作为 $2 \pi$ 的近似值. 按照阿尔. 卡西的方法, $\pi$ 的近似值的表达式是 \xparen
%     \begin{xchoices}[items = 1]
%       \item \includegraphics[width = 3cm]{example-image}
%       \item 选项
%       \item \includegraphics[width = 2cm]{example-image}
%       \item 选项
%     \end{xchoices}
%   \item 2020 年 3 月 14 日是全球首个国际圆周率日 ( $\pi$ Day). 历史上, 求圆周率 $\pi$ 的方法有多种, 与中国传统数学中的 “割圆术”相似. 数学家阿尔. 卡西的方 法是: 当正整数 $n$ 充分大时, 计算单位圆的内接正 $6 n$ 边形的周长和外切 正 $6 n$ 边形 (各边均与圆相切的正 $6 n$ 边形) 的周长, 将它们的算术平均数 作为 $2 \pi$ 的近似值. 按照阿尔. 卡西的方法, $\pi$ 的近似值的表达式是 \xparen
%     \begin{xchoices}[items = 2]
%       \item \includegraphics[width = 3cm]{example-image}
%       \item 选项
%       \item \includegraphics[width = 2cm]{example-image}
%       \item 选项
%     \end{xchoices}
%   \item 2020 年 3 月 14 日是全球首个国际圆周率日 ( $\pi$ Day). 历史上, 求圆周率 $\pi$ 的方法有多种, 与中国传统数学中的 “割圆术”相似. 数学家阿尔. 卡西的方 法是: 当正整数 $n$ 充分大时, 计算单位圆的内接正 $6 n$ 边形的周长和外切 正 $6 n$ 边形 (各边均与圆相切的正 $6 n$ 边形) 的周长, 将它们的算术平均数 作为 $2 \pi$ 的近似值. 按照阿尔. 卡西的方法, $\pi$ 的近似值的表达式是 \xparen
%     \begin{xchoices}[items = 3]
%       \item \includegraphics[width = 3cm]{example-image}
%       \item 选项
%       \item \includegraphics[width = 2cm]{example-image}
%       \item 选项 
%       \item \includegraphics[width = 2cm]{example-image}
%       \item 选项
%     \end{xchoices}
%   \item 2020 年 3 月 14 日是全球首个国际圆周率日 ( $\pi$ Day). 历史上, 求圆周率 $\pi$ 的方法有多种, 与中国传统数学中的 “割圆术”相似. 数学家阿尔. 卡西的方 法是: 当正整数 $n$ 充分大时, 计算单位圆的内接正 $6 n$ 边形的周长和外切 正 $6 n$ 边形 (各边均与圆相切的正 $6 n$ 边形) 的周长, 将它们的算术平均数 作为 $2 \pi$ 的近似值. 按照阿尔. 卡西的方法, $\pi$ 的近似值的表达式是 \xparen
%     \begin{xchoices}[items = 4]
%       \item \includegraphics[width = 3cm]{example-image}
%       \item 选项
%       \item \includegraphics[width = 2cm]{example-image}
%       \item 选项
%       \item 选项
%       \item \includegraphics[width = 3cm]{example-image}
%     \end{xchoices}
%   \item 2020 年 3 月 14 日是全球首个国际圆周率日 ( $\pi$ Day). 历史上, 求圆周率 $\pi$ 的方法有多种, 与中国传统数学中的 “割圆术”相似. 数学家阿尔. 卡西的方 法是: 当正整数 $n$ 充分大时, 计算单位圆的内接正 $6 n$ 边形的周长和外切 正 $6 n$ 边形 (各边均与圆相切的正 $6 n$ 边形) 的周长, 将它们的算术平均数 作为 $2 \pi$ 的近似值. 按照阿尔. 卡西的方法, $\pi$ 的近似值的表达式是 \xparen
%     \begin{xchoices}[items = 5]
%       \item \includegraphics[width = 3cm]{example-image}
%       \item 选项
%       \item \includegraphics[width = 2cm]{example-image}
%       \item 选项
%       \item 选项
%       \item \includegraphics[width = 3cm]{example-image}
%     \end{xchoices}
% \end{enumerate}


% \section{仅有文字}

% \begin{enumerate}
%   \item 2020 年 3 月 14 日是全球首个国际圆周率日 ( $\pi$ Day). 历史上, 求圆周率 $\pi$ 的方法有多种, 与中国传统数学中的 “割圆术”相似. 数学家阿尔. 卡西的方 法是: 当正整数 $n$ 充分大时, 计算单位圆的内接正 $6 n$ 边形的周长和外切 正 $6 n$ 边形 (各边均与圆相切的正 $6 n$ 边形) 的周长, 将它们的算术平均数 作为 $2 \pi$ 的近似值. 按照阿尔. 卡西的方法, $\pi$ 的近似值的表达式是 \xparen

%   \begin{xchoices}[items = 1]
%     \item* 正确选项
%     \item  错误选项
%     \item  错误选项
%     \item* 正确选项
%     \item* 正确选项
%     \item  错误选项
%     \item* 正确选项
%     \item* 正确选项
%   \end{xchoices}
%   \item 2020 年 3 月 14 日是全球首个国际圆周率日 ( $\pi$ Day). 历史上, 求圆周率 $\pi$ 的方法有多种, 与中国传统数学中的 “割圆术”相似. 数学家阿尔. 卡西的方 法是: 当正整数 $n$ 充分大时, 计算单位圆的内接正 $6 n$ 边形的周长和外切 正 $6 n$ 边形 (各边均与圆相切的正 $6 n$ 边形) 的周长, 将它们的算术平均数 作为 $2 \pi$ 的近似值. 按照阿尔. 卡西的方法, $\pi$ 的近似值的表达式是 \xparen

%   \begin{xchoices}[items = 2]
%     \item* 正确选项
%     \item  错误选项
%     \item  错误选项
%     \item* 正确选项
%     \item* 正确选项
%     \item  错误选项
%     \item* 正确选项
%     \item* 正确选项
%   \end{xchoices}
%   \item 2020 年 3 月 14 日是全球首个国际圆周率日 ( $\pi$ Day). 历史上, 求圆周率 $\pi$ 的方法有多种, 与中国传统数学中的 “割圆术”相似. 数学家阿尔. 卡西的方 法是: 当正整数 $n$ 充分大时, 计算单位圆的内接正 $6 n$ 边形的周长和外切 正 $6 n$ 边形 (各边均与圆相切的正 $6 n$ 边形) 的周长, 将它们的算术平均数 作为 $2 \pi$ 的近似值. 按照阿尔. 卡西的方法, $\pi$ 的近似值的表达式是 \xparen

%   \begin{xchoices}[items = 3]
%     \item* 正确选项
%     \item  错误选项
%     \item  错误选项
%     \item* 正确选项
%     \item* 正确选项
%     \item  错误选项
%     \item* 正确选项
%     \item* 正确选项
%   \end{xchoices}
%   \item 2020 年 3 月 14 日是全球首个国际圆周率日 ( $\pi$ Day). 历史上, 求圆周率 $\pi$ 的方法有多种, 与中国传统数学中的 “割圆术”相似. 数学家阿尔. 卡西的方 法是: 当正整数 $n$ 充分大时, 计算单位圆的内接正 $6 n$ 边形的周长和外切 正 $6 n$ 边形 (各边均与圆相切的正 $6 n$ 边形) 的周长, 将它们的算术平均数 作为 $2 \pi$ 的近似值. 按照阿尔. 卡西的方法, $\pi$ 的近似值的表达式是 \xparen

%   \begin{xchoices}[items = 4]
%     \item* 正确选项
%     \item  错误选项
%     \item  错误选项
%     \item* 正确选项
%     \item* 正确选项
%     \item  错误选项
%     \item* 正确选项
%     \item* 正确选项
%   \end{xchoices}
%   \item 2020 年 3 月 14 日是全球首个国际圆周率日 ( $\pi$ Day). 历史上, 求圆周率 $\pi$ 的方法有多种, 与中国传统数学中的 “割圆术”相似. 数学家阿尔. 卡西的方 法是: 当正整数 $n$ 充分大时, 计算单位圆的内接正 $6 n$ 边形的周长和外切 正 $6 n$ 边形 (各边均与圆相切的正 $6 n$ 边形) 的周长, 将它们的算术平均数 作为 $2 \pi$ 的近似值. 按照阿尔. 卡西的方法, $\pi$ 的近似值的表达式是 \xparen

%   \begin{xchoices}[items = 5]
%     \item* 正确选项
%     \item  错误选项
%     \item  错误选项
%     \item* 正确选项
%     \item* 正确选项
%     \item  错误选项
%     \item* 正确选项
%     \item* 正确选项
%   \end{xchoices}
% \end{enumerate}
% \end{document}

% \documentclass[twocolumn]{ctexart}
\documentclass{article}
\usepackage[paperwidth = 20cm]{geometry}
\usepackage{xchoices}
\usepackage{lipsum}

\begin{document}
\begin{xchoices}
  \item Lorem ipsum.
  \item orem ipsum.
\end{xchoices}
\lipsum[1]

\begin{enumerate}
  \item Lorem ipsum.
    \begin{xchoices}[]
      \item Lorem ipsum dolor sit amet, consectetuer adipiscing elit.
        Ut purus elit, vestibulum ut, placerat ac, adipiscing vitae, felis.
      \item* Lorem ipsum dolor sit amet, consectetuer adipiscing elit.
    \end{xchoices}
  \item \begin{enumerate}
    \item \begin{xchoices}
      \item Lorem ipsum dolor sit amet, consectetuer adipiscing elit.
        Ut purus elit, vestibulum ut, placerat ac, adipiscing vitae, felis.
      \item Lorem ipsum dolor sit amet, consectetuer adipiscing elit.
    \end{xchoices}
  \end{enumerate}
\end{enumerate}
Lorem ipsum dolor sit amet, consectetuer adipiscing elit.
        Ut purus elit, vestibulum ut, placerat ac, adipiscing vitae, felis.
\begin{xchoices}
  \item Lorem ipsum dolor sit amet, consectetuer adipiscing elit.
    Ut purus elit, vestibulum ut, placerat ac, adipiscing vitae, felis.
  \item Lorem ipsum dolor sit amet, consectetuer adipiscing elit.
\end{xchoices}

\lipsum[1]

\end{document}