% \documentclass{ctexart}
% \usepackage{xchoices}
% % \usepackage[a3paper]{geometry}
% \usepackage[a4paper]{geometry}

% \begin{document}


% \section{图文}

% \xchoicesetup{
%   label-pos = right-bottom,
% }
% \begin{enumerate}
%   \item 2020 年 3 月 14 日是全球首个国际圆周率日 ( $\pi$ Day). 历史上, 求圆周率 $\pi$ 的方法有多种, 与中国传统数学中的 “割圆术”相似. 数学家阿尔. 卡西的方 法是: 当正整数 $n$ 充分大时, 计算单位圆的内接正 $6 n$ 边形的周长和外切 正 $6 n$ 边形 (各边均与圆相切的正 $6 n$ 边形) 的周长, 将它们的算术平均数 作为 $2 \pi$ 的近似值. 按照阿尔. 卡西的方法, $\pi$ 的近似值的表达式是 \xparen
%     \begin{xchoices}[items = 1]
%       \item \includegraphics[width = 3cm]{example-image}
%       \item 选项
%       \item \includegraphics[width = 2cm]{example-image}
%       \item 选项
%     \end{xchoices}
%   \item 2020 年 3 月 14 日是全球首个国际圆周率日 ( $\pi$ Day). 历史上, 求圆周率 $\pi$ 的方法有多种, 与中国传统数学中的 “割圆术”相似. 数学家阿尔. 卡西的方 法是: 当正整数 $n$ 充分大时, 计算单位圆的内接正 $6 n$ 边形的周长和外切 正 $6 n$ 边形 (各边均与圆相切的正 $6 n$ 边形) 的周长, 将它们的算术平均数 作为 $2 \pi$ 的近似值. 按照阿尔. 卡西的方法, $\pi$ 的近似值的表达式是 \xparen
%     \begin{xchoices}[items = 2]
%       \item \includegraphics[width = 3cm]{example-image}
%       \item 选项
%       \item \includegraphics[width = 2cm]{example-image}
%       \item 选项
%     \end{xchoices}
%   \item 2020 年 3 月 14 日是全球首个国际圆周率日 ( $\pi$ Day). 历史上, 求圆周率 $\pi$ 的方法有多种, 与中国传统数学中的 “割圆术”相似. 数学家阿尔. 卡西的方 法是: 当正整数 $n$ 充分大时, 计算单位圆的内接正 $6 n$ 边形的周长和外切 正 $6 n$ 边形 (各边均与圆相切的正 $6 n$ 边形) 的周长, 将它们的算术平均数 作为 $2 \pi$ 的近似值. 按照阿尔. 卡西的方法, $\pi$ 的近似值的表达式是 \xparen
%     \begin{xchoices}[items = 3]
%       \item \includegraphics[width = 3cm]{example-image}
%       \item 选项
%       \item \includegraphics[width = 2cm]{example-image}
%       \item 选项 
%       \item \includegraphics[width = 2cm]{example-image}
%       \item 选项
%     \end{xchoices}
%   \item 2020 年 3 月 14 日是全球首个国际圆周率日 ( $\pi$ Day). 历史上, 求圆周率 $\pi$ 的方法有多种, 与中国传统数学中的 “割圆术”相似. 数学家阿尔. 卡西的方 法是: 当正整数 $n$ 充分大时, 计算单位圆的内接正 $6 n$ 边形的周长和外切 正 $6 n$ 边形 (各边均与圆相切的正 $6 n$ 边形) 的周长, 将它们的算术平均数 作为 $2 \pi$ 的近似值. 按照阿尔. 卡西的方法, $\pi$ 的近似值的表达式是 \xparen
%     \begin{xchoices}[items = 4]
%       \item \includegraphics[width = 3cm]{example-image}
%       \item 选项
%       \item \includegraphics[width = 2cm]{example-image}
%       \item 选项
%       \item 选项
%       \item \includegraphics[width = 3cm]{example-image}
%     \end{xchoices}
%   \item 2020 年 3 月 14 日是全球首个国际圆周率日 ( $\pi$ Day). 历史上, 求圆周率 $\pi$ 的方法有多种, 与中国传统数学中的 “割圆术”相似. 数学家阿尔. 卡西的方 法是: 当正整数 $n$ 充分大时, 计算单位圆的内接正 $6 n$ 边形的周长和外切 正 $6 n$ 边形 (各边均与圆相切的正 $6 n$ 边形) 的周长, 将它们的算术平均数 作为 $2 \pi$ 的近似值. 按照阿尔. 卡西的方法, $\pi$ 的近似值的表达式是 \xparen
%     \begin{xchoices}[items = 5]
%       \item \includegraphics[width = 3cm]{example-image}
%       \item 选项
%       \item \includegraphics[width = 2cm]{example-image}
%       \item 选项
%       \item 选项
%       \item \includegraphics[width = 3cm]{example-image}
%     \end{xchoices}
% \end{enumerate}


% \section{仅有文字}

% \begin{enumerate}
%   \item 2020 年 3 月 14 日是全球首个国际圆周率日 ( $\pi$ Day). 历史上, 求圆周率 $\pi$ 的方法有多种, 与中国传统数学中的 “割圆术”相似. 数学家阿尔. 卡西的方 法是: 当正整数 $n$ 充分大时, 计算单位圆的内接正 $6 n$ 边形的周长和外切 正 $6 n$ 边形 (各边均与圆相切的正 $6 n$ 边形) 的周长, 将它们的算术平均数 作为 $2 \pi$ 的近似值. 按照阿尔. 卡西的方法, $\pi$ 的近似值的表达式是 \xparen

%   \begin{xchoices}[items = 1]
%     \item* 正确选项
%     \item  错误选项
%     \item  错误选项
%     \item* 正确选项
%     \item* 正确选项
%     \item  错误选项
%     \item* 正确选项
%     \item* 正确选项
%   \end{xchoices}
%   \item 2020 年 3 月 14 日是全球首个国际圆周率日 ( $\pi$ Day). 历史上, 求圆周率 $\pi$ 的方法有多种, 与中国传统数学中的 “割圆术”相似. 数学家阿尔. 卡西的方 法是: 当正整数 $n$ 充分大时, 计算单位圆的内接正 $6 n$ 边形的周长和外切 正 $6 n$ 边形 (各边均与圆相切的正 $6 n$ 边形) 的周长, 将它们的算术平均数 作为 $2 \pi$ 的近似值. 按照阿尔. 卡西的方法, $\pi$ 的近似值的表达式是 \xparen

%   \begin{xchoices}[items = 2]
%     \item* 正确选项
%     \item  错误选项
%     \item  错误选项
%     \item* 正确选项
%     \item* 正确选项
%     \item  错误选项
%     \item* 正确选项
%     \item* 正确选项
%   \end{xchoices}
%   \item 2020 年 3 月 14 日是全球首个国际圆周率日 ( $\pi$ Day). 历史上, 求圆周率 $\pi$ 的方法有多种, 与中国传统数学中的 “割圆术”相似. 数学家阿尔. 卡西的方 法是: 当正整数 $n$ 充分大时, 计算单位圆的内接正 $6 n$ 边形的周长和外切 正 $6 n$ 边形 (各边均与圆相切的正 $6 n$ 边形) 的周长, 将它们的算术平均数 作为 $2 \pi$ 的近似值. 按照阿尔. 卡西的方法, $\pi$ 的近似值的表达式是 \xparen

%   \begin{xchoices}[items = 3]
%     \item* 正确选项
%     \item  错误选项
%     \item  错误选项
%     \item* 正确选项
%     \item* 正确选项
%     \item  错误选项
%     \item* 正确选项
%     \item* 正确选项
%   \end{xchoices}
%   \item 2020 年 3 月 14 日是全球首个国际圆周率日 ( $\pi$ Day). 历史上, 求圆周率 $\pi$ 的方法有多种, 与中国传统数学中的 “割圆术”相似. 数学家阿尔. 卡西的方 法是: 当正整数 $n$ 充分大时, 计算单位圆的内接正 $6 n$ 边形的周长和外切 正 $6 n$ 边形 (各边均与圆相切的正 $6 n$ 边形) 的周长, 将它们的算术平均数 作为 $2 \pi$ 的近似值. 按照阿尔. 卡西的方法, $\pi$ 的近似值的表达式是 \xparen

%   \begin{xchoices}[items = 4]
%     \item* 正确选项
%     \item  错误选项
%     \item  错误选项
%     \item* 正确选项
%     \item* 正确选项
%     \item  错误选项
%     \item* 正确选项
%     \item* 正确选项
%   \end{xchoices}
%   \item 2020 年 3 月 14 日是全球首个国际圆周率日 ( $\pi$ Day). 历史上, 求圆周率 $\pi$ 的方法有多种, 与中国传统数学中的 “割圆术”相似. 数学家阿尔. 卡西的方 法是: 当正整数 $n$ 充分大时, 计算单位圆的内接正 $6 n$ 边形的周长和外切 正 $6 n$ 边形 (各边均与圆相切的正 $6 n$ 边形) 的周长, 将它们的算术平均数 作为 $2 \pi$ 的近似值. 按照阿尔. 卡西的方法, $\pi$ 的近似值的表达式是 \xparen

%   \begin{xchoices}[items = 5]
%     \item* 正确选项
%     \item  错误选项
%     \item  错误选项
%     \item* 正确选项
%     \item* 正确选项
%     \item  错误选项
%     \item* 正确选项
%     \item* 正确选项
%   \end{xchoices}
% \end{enumerate}
% \end{document}

% \documentclass[twocolumn]{ctexart}
\documentclass{ctexart}
\usepackage{geometry}
\usepackage{amsmath}
\usepackage{xchoices}
\xchoicesetup{
  % showanswer,
  label-pos = left
}
\geometry{showframe}

\begin{document}
\begin{enumerate}
  \item 
  在直角坐标系中, 已知 $O$ 为坐标原点, $A(-1,0), B(1,0)$. 点 $P$ 满是 $k_{P A} \cdot k_{P B}=3$ 且 $|P A|+|PB|=4$, 则 $|OP|=$\xparen
      \begin{xchoices}[label-pos = left]
        \item*  $\dfrac{7 \sqrt{13}}{13}$
        \item $\dfrac{\sqrt{85}}{5}$
        \item* $\dfrac{5 \sqrt{13}}{13}$
        \item $\dfrac{\sqrt{13}}{2}$
      \end{xchoices}
  \item 
  在直角坐标系中, 已知 $O$ 为坐标原点, $A(-1,0), B(1,0)$. 点 $P$ 满是 $k_{P A} \cdot k_{P B}=3$ 且 $|P A|+|PB|=4$, 则 $|OP|=$\xparen
    \begin{xchoices}[label-pos = left]
      \item 1
      \item* 2
      \item 3
      \item* 4
    \end{xchoices}
  \item 
    在直角坐标系中, 已知 $O$ 为坐标原点, $A(-1,0), B(1,0)$. 点 $P$ 满是 $k_{P A} \cdot k_{P B}=3$ 且 $|P A|+|PB|=4$, 则 $|OP|=$\xparen
    \begin{xchoices}
      \item 1
      \item 2
      \item 3
      \item 4
    \end{xchoices}
  \item 
    在直角坐标系中, 已知 $O$ 为坐标原点, $A(-1,0), B(1,0)$. 点 $P$ 满是 $k_{P A} \cdot k_{P B}=3$ 且 $|P A|+|PB|=4$, 则 $|OP|=$\xparen
    \begin{xchoices}
      \item 1
      \item 2
      \item 3
      \item 4
    \end{xchoices}
\end{enumerate}
\end{document}