\documentclass{xdyy-usermanual}
\usepackage{xchoices}

\xdyymanualsetup{
  info = {
    author            = {夏康玮},
    title             = {用户手册},
    email             = {kangweixia_xdyy@163.com},
    date              = {2022-01-31},
    version           = {0.0.1},
    github-repository = {https://github.com/xkwxdyy/},
    gitee-repository  = {https://gitee.com/xkwxdyy/},
  }
}

\begin{document}
\maketitle
\tableofcontents

\section{宏包简介}

\pkg{xchoices}宏包是一个用于排版选项的宏包, 包含但不限于以下特点:

\begin{enumerate}
  \item 可以排版任意数量的选项
  \item 可以方便切换标签风格 \meta{label-style}
  \item 可以更改标签 \meta{label} 与内容的相对位置
  \item 可以调整标签 \meta{label} 的偏移
\end{enumerate}


\section{宏包开发背景}


已经存在用 \href{https://www.latexstudio.net/index/details/index/mid/2270.html}{\pkg{ifthen}宏包} 或者用 \href{https://www.latexstudio.net/index/details/index/mid/2191.html}{\pkg{xcoffins}宏包} 写的相关选项命令, 用于排版试卷中的选择题, 常见的形如 \tn{xx\marg{arg1}\marg{arg2}\marg{arg3}\marg{arg4}}, 但是有几个不足(以 \tn{xx} 命令为例):
\begin{enumerate}
  \item 从用户接口角度看:这样定义的命令只能且必须接受四个参数
    \begin{itemize}
      \item 如果输入少于四个, 那么就会有空白项出现, 比如\verb*|D.  |
      \item 如果想要输入多于四个固然可以用同样的方式再定义相应的命令, 但是可能事先并不知道有多少个选项, 通常的解决办法是先建立很多个命令分别作用于$1, 2, \cdots, 9$个命令, 比较麻烦不够简洁, 而且问题又来了: 通常的LaTeX命令参数最多有9个, 如果有排版更多项的需求时, 传统的方法一般是通过多个命令嵌套实现,但是无论是代码实现还是用户接口都不令人满意, 十分繁琐!
      
      \emph{所以希望存在一个相同接口的命令或环境, 可以排版任意个选项.}
    \end{itemize}
  \item 从代码实现角度看, 已有命令的代码在实现上有很大的优化空间
  \item 从功能的延拓性看:现有的一些命令中,“ABCD”往往是手动输入进行封装, 对 \meta{label} 样式切换造成了困难。而在选项排版中,不同的 \meta{label}  样式的方便切换( 比如 arabic, roman等)也是一个大需求,所以需要在排出来的基础上提供方便的 \kvopt{\meta{key}}{\meta{val}} 接口来进行控制
\end{enumerate}

\end{document}

\documentclass{xdyy-usermanual}
\usepackage{xchoices}

\xdyymanualsetup{
  info = {
    author            = {夏康玮},
    title             = {用户手册},
    email             = {kangweixia_xdyy@163.com},
    date              = {2022-01-31},
    version           = {0.0.1},
    github-repository = {https://github.com/xkwxdyy/},
    gitee-repository  = {https://gitee.com/xkwxdyy/},
  }
}

\begin{document}
\maketitle
\tableofcontents






\section{用户接口}


% \subsection{主要命令}

\begin{function}[added = 2021-12-18,updated = 2021-12-25]{\choices, \choices*}
  \begin{syntax}
    |\choices| \oarg{键值列表} \marg{内容}
    |\choices*| \oarg{键值列表} \marg{内容}
  \end{syntax}
  \marg{内容} 中不同选项用 \verb|&&| 分隔, 且必须是两个, 只用一个会报错. 有两点需要注意: 
    \begin{itemize}
      \item 之所以用两个 \& 作为分隔符是考虑到可能出现使用 \env{align} 或 \env{tabular} 等需要使用\&的环境的情况, 如果使用一个 \& 作为分隔符可能会“误伤”
      \item 在选项中可正常通过 \verb|\&| 排版 \& 符号
      \item 在内容的结尾是否添加 \verb|&&|都不会造成空白项(会经过函数过滤)
    \end{itemize}
\end{function}

% \begin{function}[added = 2021-12-23]{\choicesetup}
%   \begin{syntax}
%     |\choicesetup| \marg{键值列表}
%   \end{syntax}
%   宏包相关键值的设置, 影响使用该命令后面的\tn{choices(*)}命令的相关初始值. 详细键值说明见 \ref{subsec:键值说明}.
% \end{function}



% \subsection{辅助命令}

% \begin{function}[added = 2021-12-18, updated = 2021-12-22]{\hlistchoice}
%   \begin{syntax}
%     |\hlistchoice| \oarg{键值列表} \marg{内容}
%   \end{syntax}
%   基于 \env{hlist} 环境写的选项排版命令, 可直接使用, 等效于\tn{choices}.
% \end{function}

% \begin{function}[added = 2021-12-21, updated = 2021-12-22]{\coffinchoice}
%   \begin{syntax}
%     |\coffinchoice| \oarg{键值列表} \marg{内容}
%   \end{syntax}
%   基于 \LaTeX3 的 \pkg{coffin} 模块写的选项排版命令, 可直接使用, 等效于\tn{choices*}.
% \end{function}

% \begin{function}[added = 2021-12-21]{\quan}
%   \begin{syntax}
%     |\quan| \marg{number}
%   \end{syntax}
%   基于 \pkg{tikz} 宏包绘制的带圈数字命令, 根据数字大小判断进行水平垂直方向的压缩, 可单独使用. 
% \end{function}


% \subsection{\pkg{choices}宏包相关的键值说明}\label{subsec:键值说明}
% 若出现一个 \meta{key} 下面有相类似的\meta{key}, 比如 \cmd{colsep} 与 \cmd{col-sep}, 是笔者为了不增加用户的记忆负担, 设置了多个等效键值, 甚至用户可以在 \file{choice.sty} 中修改源码, 仿照作者的做法, 添加自己喜欢的键值, 不过最好有一定 \LaTeX3 基础, 当然也欢迎联系作者\footnote{kangweixia_xdyy@163.com}、到仓库中\href{https://github.com/xkwxdyy/choice-l3/issues}{提issue} 等多种方式与作者联系. 

% \subsubsection{\tn{choices}与\tn{choices*}作用均有效的键值}
% \begin{function}[updated = 2022-01-11]{label-style}
%   \begin{syntax}
%     label-style = \meta{arabic, alph, Alph, roman, Roman, quan, chinese} \init{Alph}
%   \end{syntax}
%   设置标签的风格: 
%   \begin{itemize}
%     \item arabic: 阿拉伯数字
%     \item alph: 小写英文
%     \item Alph: 大写英文
%     \item roman: 小写罗马数字
%     \item Roman: 大写罗马数字
%     \item quan: 带圈数字
%     \item chinese: 中文数字
%     % \item none: 没有计数的空白效果
%   \end{itemize}
% \end{function}


% \begin{function}[updated = 2022-01-09]{items}
%   \begin{syntax}
%     |items| = \meta{number}
%   \end{syntax}
%   手动设置每行排多少项(否则会根据选项宽度自动排版)
% \end{function}

% \begin{function}[updated = 2022-01-09]{pre-label}
%   \begin{syntax}
%     |pre-label| = \marg{sth placed before label} \init{{}}
%   \end{syntax}
%   标签后的相关设置, 类似于 \pkg{hlist} 宏包的 \cmd{pre label}, 默认是空. 
% \end{function}

% \begin{function}[updated = 2022-01-09]{post-label}
%   \begin{syntax}
%     |post-label| = \marg{sth placed after label} \init{{.}}
%   \end{syntax}
%   标签后的相关设置, 类似于 \pkg{hlist} 宏包的 \cmd{post label}, 默认是加一个点, 产生效果为|A.|
%   \cmd{label-style} 设置为 \cmd{quan} 的时候会默认把 \cmd{poslabel}的点去掉, 符合主流习惯.
% \end{function}

% \begin{function}[added = 2021-12-25, updated = 2022-01-09]{random-items}
%   \begin{syntax}
%     |random-items| = \meta{true, false} \init{false}
%   \end{syntax}
%   控制选项是否进行乱序显示, 此键值只能通过 \tn{choicesetup} 使用, 作用范围为更改后的所有命令.
% \end{function}



% \subsubsection{仅对\tn{choices*}产生效果的键值}
% \remark{
%   下面所说的键值如果作用在\tn{choices}上并不会报错, 但不会产生作用, 这么设计是为了方便用户在\tn{choices}与\tn{choices*}两个命令之间自由切换而不用考虑这个键值.
% }
% \begin{function}[updated = 2021-12-23]{anchor}
%   \begin{syntax}
%     |anchor| = \meta{方位} \init{west}
%   \end{syntax}
%   标签与内容的相对位置(有\pkg{tikz}基础的用户容易理解, 其他用户可以设置不同的 \cmd{anchor} 查看效果也能很快理解)
%   \begin{itemize}
%     \item north
%     \item south
%     \item east
%     \item west
%     \item northwest
%     \item northeast
%     \item southeast
%     \item southwest
%   \end{itemize}
% \end{function}


% \begin{function}[added = 2021-12-25, updated = 2022-01-09]{autopic-anchor}
%   \begin{syntax}
%     |autopic-anchor| = \meta{方位} \init{south}
%   \end{syntax}
%   \pkg{choices} 宏包设置了检测是否内容中包含 \tn{includegraphics} 命令并自动进行 \kvopt{\meta{anchor}}{\meta{方位}} 的设置, 默认是\kvopt{\meta{anchor}}{\meta{south}}, 更改 \cmd{autopic-anchor} 的值可更改默认方位.
%   注意这个键值要使用 \tn{choicesetup} 进行更改, 如
%   \lstset{ linewidth = 0.4\linewidth }
%   \begin{LaTeXdemo}
%     \choicesetup{
%       autopic-anchor = north
%     }
%   \end{LaTeXdemo}
%   作用范围为在所有更改后的命令上.
% \end{function}
% \begin{function}{align}
%   \begin{syntax}
%     |align| = \meta{left, center, right} \init{center}
%   \end{syntax}
%   选项的对齐方式:
%   \begin{itemize}
%     \item left: 左对齐
%     \item center: 居中
%     \item right: 右对齐
%   \end{itemize}
% \end{function}


% \begin{function}{xshift}
%   \begin{syntax}
%     |xshift| = \meta{dimension}
%   \end{syntax}
%   % 在不修改 \cmd{align} (即默认使用\kvopt{\meta{align}}{\meta{center}})的情况下, 增加列之间的距离, 与下方的 \cmd{colsep} 的区别是使用 \cmd{xshift} 整体会偏移
%   整体的水平偏移量.
% \end{function}

% \begin{function}[updated = 2021-12-25]{yshift}
%   \begin{syntax}
%     |yshift| = \meta{dimension}
%   \end{syntax}
%   整体的垂直偏移量.
% \end{function}

% \begin{function}[added = 2021-12-25, updated = 2022-01-09]{below-sep}
%   \begin{syntax}
%     |below-sep| = \meta{dimension}
%   \end{syntax}
%   \cmd{item} 的垂直下方偏移量, 每一项都会作用. 可以形成“每个选项下方都有留白”的效果, 可以用作给学生留白书写等.
%   % 如果想要整体上下移动, 请使用 \cmd{topsep} 和 \cmd{bottomsep}.
% \end{function}

% \begin{function}{label-xshift}
%   \begin{syntax}
%     |label-xshift| = \meta{dimension}
%   \end{syntax}
%   标签 \cmd{label} 的水平偏移量.
% \end{function}

% \begin{function}{label-yshift}
%   \begin{syntax}
%     |label-yshift| = \meta{dimension}
%   \end{syntax}
%   标签 \cmd{label} 的垂直偏移量.
% \end{function}

% \begin{function}[updated = 2022-01-09]{top, top-sep}
%   \begin{syntax}
%     |top| = \meta{dimension}
%     |top-sep| = \meta{dimension}
%   \end{syntax}
%   整体与上方内容的垂直偏移量.
% \end{function}


% \begin{function}[updated = 2022-01-09]{bottom, bottom-sep}
%   \begin{syntax}
%     |bottom| = \meta{dimension}
%     |bottom-sep| = \meta{dimension}
%   \end{syntax}
%   整体与下方内容的垂直偏移量.
% \end{function}


% \begin{function}[updated = 2022-01-09]{row-sep}
%   \begin{syntax}
%     |row-sep| = \meta{dimension}
%   \end{syntax}
%   第一行保持不动, 下方行之间额外的垂直间距偏移.
% \end{function}

% \begin{function}[updated = 2022-01-09]{col-sep, item-sep}
%   \begin{syntax}
%     |col-sep| = \meta{dimension}
%     |item-sep| = \meta{dimension}
%   \end{syntax}
%     \begin{itemize}
%       \item 在不修改 \cmd{align} (即默认使用\kvopt{\meta{align}}{\meta{center}})的情况下, 列之间额外的水平间距偏移(效果类似于 \tn{hfill} 的感觉). 
%       \item \kvopt{\meta{align}}{\meta{left}} 的时候保持第一列不动, 列之间额外的水平间距偏移. 
%     \end{itemize}
% \end{function}
\end{document}